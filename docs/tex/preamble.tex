\usepackage{cmap} % Улучшенный поиск русских слов в полученном pdf-файле
\usepackage[T2A]{fontenc} % Поддержка русских букв
\usepackage[utf8]{inputenc} % Кодировка utf8
\usepackage[english,russian]{babel} % Языки: русский, английский

\usepackage[final]{pdfpages}

\usepackage[14pt]{extsizes}

\usepackage{indentfirst}
\setlength{\parindent}{1.25cm}

\usepackage{multirow}
\usepackage{threeparttable}

\usepackage{csvsimple}
\usepackage{geometry}
\geometry{left=30mm}
\geometry{right=10mm}
\geometry{top=20mm}
\geometry{bottom=20mm}

\usepackage{titlesec}
\titleformat{\section}
{\normalsize\bfseries}
{\thesection}
{1em}{}
\titlespacing*{\chapter}{0pt}{-30pt}{8pt}
\titlespacing*{\section}{\parindent}{*4}{*4}
\titlespacing*{\subsection}{\parindent}{*4}{*4}
\titleformat{\chapter}{\LARGE\bfseries}{\thechapter}{16pt}{\LARGE\bfseries}
\titleformat{\section}{\Large\bfseries}{\thesection}{16pt}{\Large\bfseries}

\usepackage{setspace}
\onehalfspacing % Полуторный интервал

\sloppy

\frenchspacing
\usepackage{indentfirst} % Красная строка

\usepackage{amsmath} % Для титульника
\usepackage{enumitem} 
\setenumerate[0]{label=\arabic*)} % Изменение вида нумерации списков
\renewcommand{\labelitemi}{---}

\usepackage{caption}
\captionsetup[table]{justification=raggedright,singlelinecheck=off} % Изменение подписей к таблицам
\captionsetup{labelsep=endash, justification=centering} % Настройка подписей float объектов
\captionsetup[figure]{name=Рисунок} % Изменение подписей к рисункам

\addto\captionsrussian{\renewcommand{\bibname}{Список использованных источников}}

\makeatletter 
\def\@biblabel#1{#1. } % Изменение нумерации списка использованных источников

% Forbid hyphenation in sections
\renewcommand\section{\@startsection {section}{1}{\z@}%
	{-3.5ex \@plus -1ex \@minus -.2ex}%
	{2.3ex \@plus.2ex}%
	{\normalfont\Large\bfseries\sloppy}}
% Forbid line split in subsections
\renewcommand\subsection{\@startsection{subsection}{2}{\z@}%
	{-3.25ex\@plus -1ex \@minus -.2ex}%
	{1.5ex \@plus .2ex}%
	{\normalfont\large\bfseries\hbox}}

\makeatother

\usepackage{listings} % Для листинга кода
\lstset{ %
	language=sql,
	basicstyle=\small\sffamily,			% размер и начертание шрифта для подсветки кода
	numbers=none,						% где поставить нумерацию строк (слева\справа)
	numberstyle=\tiny,		     		% размер шрифта для номеров строк
	stepnumber=1,						% размер шага между двумя номерами строк
	numbersep=5pt,						% как далеко отстоят номера строк от подсвечиваемого кода
	frame=single,						% рисовать рамку вокруг кода
	tabsize=4,							% размер табуляции по умолчанию равен 4 пробелам
	captionpos=t,						% позиция заголовка вверху [t] или внизу [b]
	breaklines=true,					
	breakatwhitespace=true,				% переносить строки только если есть пробел
	backgroundcolor=\color{white},
	basicstyle=\footnotesize\ttfamily,
	keywordstyle=\color{blue},
	stringstyle=\color{red},
	commentstyle=\color{gray},
	showspaces=false,
	showstringspaces=false
}

\usepackage{pgfplots} % для графиков
\pgfplotsset{compat=1.9}
\usetikzlibrary{datavisualization}
\usetikzlibrary{datavisualization.formats.functions}

\usepackage{graphicx}
\newcommand{\img}[3] {
	\begin{figure}[h!]
		\center{\includegraphics[height=#1]{img/#2}}
		\caption{#3}
		\label{img:#2}
	\end{figure}
}

\usepackage[unicode,pdftex]{hyperref} % Ссылки в pdf
\hypersetup{hidelinks}





