\chapter{Исследовательская часть}

\section{Описание исследования}

Индекс -- это объект базы данных, предназначенный для сокращения времени выполнения запроса и обеспечивающий быстрый доступ к данным.
Индексы являются указателями, которые позволяют эффективно находить и извлекать записи, не прибегая к полному сканированию таблицы.
Они играют ключевую роль в оптимизации работы баз данных при обработке больших объемов информации.

Было принято решение провести исследование зависимости времени запроса от количества записей в базе данных при наличии и отсутствии дополнительных индексов.
Для исследования влияния индекса на время выполнения запроса поиска создан индекс на столбцах ($doctor\_id$, $patient\_id$) таблицы Appointments.

Исследовался запрос, приведенный в листинге \ref{lst:query}

\begin{lstlisting}[label=lst:query,caption=\raggedright{Исследованный запрос}]
SELECT * FROM appointments
WHERE doctorid = 200 and patientid = 200;
\end{lstlisting}

Время выполнения запроса в зависимости от количества записей в таблице без дополнительных индексов приведено в таблице~\ref{table:noindextime}.

\begin{table}[h]
	\begin{center}
		\caption{Замер времени при отсутствии дополнительных индексов}
		\label{table:noindextime}
		\begin{tabular}{|c|c|}
			\hline
			\bfseries Количество записей & \bfseries  Время запроса, мкс
			\csvreader{csv/noindex.csv}{}
			{\\\hline \csvcoli&\csvcolii}
			\\\hline
		\end{tabular}
	\end{center}
\end{table}

\clearpage
Время выполнения запроса в зависимости от количества записей в таблице с использованием дополнительных индексов приведено в таблице~\ref{table:indextime}.

\begin{table}[h]
	\begin{center}
		\captionsetup{justification=raggedright,singlelinecheck=off}
		\caption{Замер времени при наличии дополнительных индексов}
		\label{table:indextime}
		\begin{tabular}{|c|c|}
			\hline
			\bfseries Количество записей & \bfseries  Время запроса, мкс
			\csvreader{csv/index.csv}{}
			{\\\hline \csvcoli&\csvcolii}
			\\\hline
		\end{tabular}
	\end{center}
\end{table}

На рисунке~\ref{img:g1} представлен график, иллюстрирующий зависимость времени выполнения запроса от количества записей в таблице при наличии и отсутствии дополнительных индексов.

\clearpage
\begin{figure}[h!]
	\centering
	\begin{tikzpicture}
		\begin{axis}[	
			xmin=75,
			xmode=log,
			height = 0.375\paperheight, 
			width = 0.609375\paperwidth,
			xmin=-50,
			legend pos = north west,
			table/col sep=comma,
			xlabel={количество записей},
			ylabel={время, мкс},
			]
			\legend{ 
				Без индекса, 
				С индексом,
			};
			\addplot [
			solid, 
			draw = blue,
			mark = *, 
			mark options = {
				scale = 1.5, 
				fill = blue, 
				draw = black
			}
			] table [x={count}, y={time}] {csv/noindex.csv};
			\addplot [
			dashed, 
			draw = brown,	
			mark = star, 
			mark options = {
				scale = 1.5, 
				draw = brown
			}
			] table [x={count}, y={time}] {csv/index.csv};
		\end{axis}
	\end{tikzpicture}
	\caption{Сравнение времени выполнения запроса поиска при отсутствии и наличии доп. индекса}
	\label{img:g1}
\end{figure}

В результате анализа данных, представленных в таблицах \ref{table:noindextime} и \ref{table:indextime}, сделан следующий вывод: скорость выполнения запроса поиска в таблице с дополнительным индексом выше, чем скорость выполнения этого запроса в таблице без индекса.

В таблице \ref{table:comparing} приведен результат сравнения времени выполнения запроса в таблицах с дополнительным индексом и без него при 1000, 10000, 25000, 50000 и 100000 записей:

\begin{table}[h]
	\begin{center}
		\captionsetup{justification=raggedright,singlelinecheck=off}
		\caption{Результат сравнения при 1000, 10000, 25000, 50000 и 100000 записей в таблицах}
		\label{table:comparing}
		\begin{tabular}{|c|c|}
			\hline
			Количество записей & $t_{\text{без индекса}} / t_{\text{с индексом}}$ \\ \hline
			1000               & 1.68               \\ \hline
			10000              & 1.51               \\ \hline
			25000 			   & 1.49				\\ \hline 
			50000              & 1.48               \\ \hline
			100000             & 1.38               \\ \hline
		\end{tabular}
	\end{center}
\end{table}

\clearpage
Таким образом, время выполнения запроса в таблице без дополнительного индекса, содержащей 100000 записей, в 3.06 раз выше, чем время выполнения запроса в такой таблице, содержащей 10 записей.
В таблице с дополнительным индексом при том же количестве записей значение данного коэффициента равно 2.61.

\section*{Вывод}
Из результатов проведенного исследования следует вывод: выполнение запроса в таблицу без дополнительных индексов требует в среднем на 42~\% больше времени, чем выполнение того же запроса в таблице с индексом.

