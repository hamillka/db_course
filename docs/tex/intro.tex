\begin{center}
	\textbf{\large ВВЕДЕНИЕ}
\end{center}
\addcontentsline{toc}{chapter}{ВВЕДЕНИЕ}

Люди регулярно сталкиваются с необходимостью записаться на прием к врачу.
Это может быть связано с различными обстоятельствами: от необходимости срочного медицинского вмешательства в случае травмы до планового профилактического обследования. 
Независимо от причины, каждый человек хочет, чтобы процесс записи прошел быстро и без лишних затруднений, а выбранный врач обладал высокой квалификацией и помог решить проблему со здоровьем.

В условиях современного мира, где технологии проникают во все сферы жизни, становится особенно важно сделать процесс записи к врачу максимально удобным и доступным.
Традиционные способы, такие как телефонные звонки или личное посещение поликлиники, все чаще вызывают неудобства. 
Ожидание на линии, ограниченные часы работы регистратуры, необходимость выделять время на визит -- все это добавляет стресса и усложняет жизнь пациентам.

Кроме того, неэффективное управление потоками пациентов приводит к перегрузке медицинского персонала и снижает качество предоставляемых услуг. 
В связи с этим растет потребность в разработке удобных систем, которые упростят процесс записи на прием. 
Цифровые платформы и специализированные приложения, предназначенные для этих целей, могут существенно изменить ситуацию в лучшую сторону. 
Они позволяют пациентам самостоятельно выбирать удобное время и дату визита, а также дают доступ к информации о врачах, что способствует более осознанному выбору специалиста.

Внедрение таких технологий не только экономит время, но и улучшает взаимодействие пациентов и врачей. 
Удобство записи и возможность заранее подготовиться к приему повышают эффективность медицинской помощи. 
В результате пациенты реже откладывают визит к врачу, а это, в свою очередь, способствует своевременному выявлению и лечению заболеваний.

Таким образом, внедрение современных технологий в процесс записи к врачу имеет значительные преимущества как для пациентов, так и для медицинских учреждений. 
Это важный шаг к созданию более доступной и качественной системы здравоохранения, которая отвечает требованиям сегодняшнего дня.

\clearpage
Целью курсовой работы является разработка базы данных для приложения сети медицинских клиник. 

Для достижения поставленной цели необходимо решить следующие задачи:
\begin{itemize}
	\item провести анализ существующих решений;
	\item формализовать задачу и определить функционал веб-приложения;
	\item провести анализ моделей баз данных и выбрать наиболее подходящую;
	\item провести анализ существующих СУБД и выбрать наиболее подходящую;
	\item спроектировать и разработать базу данных;
	\item спроектировать и разработать Web-приложение;
	\item исследовать зависимость времени запроса от количества записей в базе данных при наличии и отсутствии дополнительных индексов в базе.
\end{itemize}
