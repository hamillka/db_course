\begin{center}
	\textbf{\large ЗАКЛЮЧЕНИЕ}
\end{center}
\refstepcounter{chapter}
\addcontentsline{toc}{chapter}{ЗАКЛЮЧЕНИЕ}

В рамках курсовой работы было разработано программное обеспечение для взаимодействия с базой данных сети медицинских клиник, позволяющее получить доступ к информации о записях, врачах и пациентах и изменить ее.

При увеличении количества записей выполнение запроса в таблице с дополнительным индексом эффективнее примерно на 17.2~\%.

В ходе выполнения курсовой работы были решены все поставленные задачи:
\begin{itemize}
	\item проведен анализ существующих решений;
	\item формализована задача и определен функционал веб-приложения;
	\item проведен анализ моделей баз данных и выбрана наиболее подходящая;
	\item проведен анализ существующих СУБД и выбрана наиболее подходящая;
	\item спроектирована и разработана база данных;
	\item спроектировано и разработано Web-приложение;
	\item проведено исследование зависимости времени запроса от количества записей в базе данных при наличии и отсутствии дополнительных индексов в базе.
\end{itemize} 

