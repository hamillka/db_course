\chapter{Технологическая часть}

\section{Средства реализации}
В данной работе для написания Web-приложения был выбран язык программирования $Golang$~\cite{golang} ввиду следующих причин:
\begin{itemize}
	\item компиляция в машинный код: Go компилируется непосредственно в машинный код, что обеспечивает высокую скорость выполнения приложений;
	\item эффективное использование памяти: Go оптимизирует использование памяти, что позволяет веб-приложениям работать более эффективно~\cite{goperf};
	\item обнаружение ошибок на этапе компиляции: статическая типизация позволяет выявлять ошибки типов до выполнения программы, что повышает надежность и безопасность кода;
	\item явное управление типами: явное указание типов данных улучшает читаемость и понимание кода;
	\item широкий набор стандартных библиотек: Go поставляется с обширной стандартной библиотекой, включающей поддержку работы с HTTP, JSON, базами данных, шаблонами;
	\item инструменты для тестирования и профилирования: встроенные инструменты для тестирования, профилирования и анализа кода помогают улучшить качество разработки;
	\item кросс-компиляция: Go легко компилируется для различных платформ, что облегчает создание приложений для различных операционных систем.

\end{itemize}

В качестве среды разработки был выбран $JetBrains$ $Goland$~\cite{goland} по следующим причинам:
\begin{itemize}
	\item интеграция с lint-инструментами: поддержка GoLint и других инструментов для статического анализа, которые могут быть интегрированы для улучшения качества кода;
	\item поддержка Git: полная интеграция с системой контроля версий, включая инструменты для управления ветками, слияний и просмотра истории коммитов;
	\item мощный отладчик: GoLand предоставляет инструменты для пошаговой отладки, установки точек останова и анализа состояния программы во время выполнения;
	\item интеграция с Docker: поддержка создания, управления и отладки Docker-контейнеров прямо из IDE;
	\item мульти-языковая поддержка: GoLand поддерживает HTML, CSS и TypeScript, что удобно для разработки веб-приложений;
	\item поддержка фреймворков и библиотек: интеграция с фреймворками и библиотеками для веб-разработки, такими как React, Angular и Vue.js.

\end{itemize}

При выборе средства для разработки пользовательского интерфейса был выбран фреймворк $TypeScript$ $React$~\cite{tsreact} по следующим причинам:
\begin{itemize}
	\item статическая типизация: TypeScript позволяет определить типы данных, что помогает предотвратить ошибки на этапе компиляции;
	\item раннее обнаружение ошибок: TypeScript позволяет находить ошибки типов и синтаксиса на этапе компиляции, до выполнения кода, что уменьшает количество ошибок при выполнении;
	\item подробная документация: React достаточно полно описан в официальной документации, что упрощает его изучение.
\end{itemize}

\newpage
\section{Выбор СУБД}

Выбор СУБД проводился среди MSSQL, Oracle, PostgreSQL и MySQL.

В таблице \ref{tbl:analysis} приведен сравнительный анализ рассматриваемых СУБД, который проводился по следующим критериям:
\begin{enumerate}
	\item бесплатное распространение СУБД;
	\item возможность создания ролевой модели;
	\item наличие опыта работы с СУБД.
\end{enumerate}

\begin{table}[!ht]
	\begin{center}
		\caption{Сравнительный анализ СУБД}
		\label{tbl:analysis}
		\begin{tabular}{|c|c|c|c|c|}
			\hline
			Критерий & MSSQL & Oracle & PostgreSQL & MySQL \\ \hline
			1        & -     & -      & +          & +     \\ \hline
			2        & +     & +      & +          & +     \\ \hline
			3        & -     & -      & +          & -     \\ \hline
		\end{tabular}
	\end{center}
\end{table}

В результате сравнительного анализа было принято решение использовать СУБД PostgreSQL~\cite{psql}, поскольку она удовлетворяет всем обозначенным критериям.

\section{Реализация хранимой функции, таблиц и ролей базы данных}
В листинге~\ref{lst:proc} приложения Б приведена реализация создания хранимой функции $add\_appointment()$ для добавления записи в таблицу Appointments.

В листинге~\ref{lst:init} приложения Б приведена реализация создания таблиц базы данных.

В листинге~\ref{lst:roles} приложения Б приведена реализация создания ролей уровня базы данных.

\clearpage
\section{Тестирование хранимой функции}

Для тестирования хранимой функции $add\_appointment()$, реализация которой представлена в листинге~\ref{lst:proc}, было принято решение использовать язык программирования $Python$ и библиотеку $psycopg2$~\cite{psycopg}, которая предоставляет интерфейс для работы с базой данных.

Написанный скрипт выполняет следующие действия:
\begin{itemize}
	\item создает подключение к базе данных с тестируемой функцией;
	\item вызывает хранимую функцию, которая выполняет добавление новой записи в таблицу Appointments;
	\item сравнивает содержимое таблицы Appointments до и после вызова функции.
\end{itemize}

В листинге~\ref{lst:testing} приложения Б приведена реализация описанного выше скрипта для тестирования хранимой функции.

\section{Автоматизация развертывания}

Для автоматизации развертывания приложения и управления им использованы инструменты Docker и Docker Compose.

Docker -- это платформа для автоматизации развертывания приложений в контейнерах, обеспечивающая изоляцию приложения и его зависимостей от системы. 
Контейнеры Docker позволяют гарантировать, что приложение будет работать одинаково на всех окружениях -- от локальной разработки до тестовых и производственных серверов.

Docker Compose -- это инструмент для определения и управления многоконтейнерными Docker-приложениями. 
С помощью Docker Compose можно описать всю инфраструктуру приложения, включая базы данных, в одном файле и запускать их одной командой.

Для удобства развертывания и управления зависимостями в рамках проекта был создан файл docker-compose.yml, который включает описание сервисов базы данных и основного приложения.

Листинг описанного выше файла приведен в листинге~\ref{lst:dcompose}.

