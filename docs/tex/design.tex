\chapter{Конструкторская часть}

\section{Проектирование базы данных}

Ниже приведены таблицы базы данных:
\begin{itemize}
	\item doctors -- таблица врачей;
	\item patients -- таблица пациентов;
	\item offices -- таблица кабинетов;
	\item branches -- таблица филиалов;
	\item medical\_histories -- таблица медицинских карт пациентов;
	\item users -- таблица пользователей;
	\item appointments -- таблица-связь врачей и пациентов;
	\item timetable -- таблица-связь врачей и кабинетов.
\end{itemize}

Диаграмма базы данных представлена на рисунке \ref{img:diagram}.

\img{95mm}{diagram}{Диаграмма базы данных}

\clearpage
В таблицах 2.1--\ref{table:timetable} приведено описание атрибутов каждой таблицы и их ограничения целостности.

\begin{table}[!ht]
	\begin{center}
		\label{tbl:doctors}
		\caption{Описание таблицы Doctors}
		\begin{tabular}{|c|c|c|c|}
			\hline
			\textbf{Атрибут} & \textbf{Назначение}                                               & \textbf{Тип данных} & \textbf{Ограничение}                                                 \\ \hline
			id                & \begin{tabular}[c]{@{}c@{}}Идентификатор\\ врача\end{tabular}     & Целое число         & \begin{tabular}[c]{@{}c@{}}Первичный ключ,\\ не нулевой\end{tabular} \\ \hline
			fio               & ФИО врача                                                         & Строка              & Не нулевой                                                           \\ \hline
			phoneNumber       & \begin{tabular}[c]{@{}c@{}}Номер телефона \\ врача\end{tabular}   & Строка              & Не нулевой                                                           \\ \hline
			email             & \begin{tabular}[c]{@{}c@{}}Электронная почта\\ врача\end{tabular} & Строка              & Не нулевой                                                           \\ \hline
			specialization             & \begin{tabular}[c]{@{}c@{}}Специализация\\ врача\end{tabular} & Строка              & Не нулевой                                                           \\ \hline
		\end{tabular}
	\end{center}
\end{table}

\begin{table}[!ht]
	\begin{center}
		\label{table:patients}
		\caption{Описание таблицы Patients}
		\begin{tabular}{|c|c|c|c|}
			\hline
			\textbf{Атрибут} & \textbf{Назначение}                                               & \textbf{Тип данных} & \textbf{Ограничение}                                                 \\ \hline
			id                & \begin{tabular}[c]{@{}c@{}}Идентификатор\\ пациента\end{tabular}     & Целое число         & \begin{tabular}[c]{@{}c@{}}Первичный ключ,\\ не нулевой\end{tabular} \\ \hline
			fio               & ФИО пациента                                                         & Строка              & Не нулевой                                                           \\ \hline
			phoneNumber       & \begin{tabular}[c]{@{}c@{}}Номер телефона \\ пациента\end{tabular}   & Строка              & Не нулевой                                                           \\ \hline
			email             & \begin{tabular}[c]{@{}c@{}}Электронная почта\\ пациента\end{tabular} & Строка              & Не нулевой                                                           \\ \hline
			insurance 		& \begin{tabular}[c]{@{}c@{}}Номер страховки\\ пациента\end{tabular} & Строка & Не нулевой \\ \hline
		\end{tabular}
	\end{center}
\end{table}

\begin{table}[!ht]
	\begin{center}
		\label{table:branches}
		\caption{Описание таблицы Branches}
		\begin{tabular}{|c|c|c|c|}
			\hline
			\textbf{Атрибут} & \textbf{Назначение}                                               & \textbf{Тип данных} & \textbf{Ограничение}                                                 \\ \hline
			id                & \begin{tabular}[c]{@{}c@{}}Идентификатор\\ филиала\end{tabular}     & Целое число         & \begin{tabular}[c]{@{}c@{}}Первичный ключ,\\ не нулевой\end{tabular} \\ \hline
			name               & Название филиала                                                         & Строка              & Не нулевой                                                           \\ \hline
			phoneNumber       & \begin{tabular}[c]{@{}c@{}}Номер телефона \\ ресепшена\end{tabular}   & Строка              & Не нулевой                                                           \\ \hline
			address             &  Адрес филиала & Строка              & Не нулевой                                                           \\ \hline
		\end{tabular}
	\end{center}
\end{table}

\begin{table}[!ht]
	\begin{center}
		\label{table:offices}
		\caption{Описание таблицы Offices}
		\begin{tabular}{|c|c|c|c|}
			\hline
			\textbf{Атрибут} & \textbf{Назначение}                                               & \textbf{Тип данных} & \textbf{Ограничение}                                                 \\ \hline
			id                & \begin{tabular}[c]{@{}c@{}}Идентификатор\\ кабинета\end{tabular}     & Целое число         & \begin{tabular}[c]{@{}c@{}}Первичный ключ,\\ не нулевой\end{tabular} \\ \hline
			number               & Номер кабинета                                                         & Целое число              & Не нулевой                                                           \\ \hline
			floor       & Этаж  & Целое число              & Не нулевой                                                           \\ \hline
			branchID             &   \begin{tabular}[c]{@{}c@{}}Идентификатор \\филиала, в котором \\расположен кабинет\end{tabular} & Целое число              & \begin{tabular}[c]{@{}c@{}}Внешний ключ,\\ не нулевой\end{tabular}                                                           \\ \hline
		\end{tabular}
	\end{center}
\end{table}

\begin{table}[!ht]
	\begin{center}
		\label{table:medicalhistories}
		\caption{Описание таблицы MedicalHistories}
		\begin{tabular}{|c|c|c|c|}
			\hline
			\textbf{Атрибут} & \textbf{Назначение}                                               & \textbf{Тип данных} & \textbf{Ограничение}                                                 \\ \hline
			id                & \begin{tabular}[c]{@{}c@{}}Идентификатор\\ карты\end{tabular}     & Целое число         & \begin{tabular}[c]{@{}c@{}}Первичный ключ,\\ не нулевой\end{tabular} \\ \hline
			chronic\_diseases               & \begin{tabular}[c]{@{}c@{}}Хронические\\ заболевания\\ пациента\end{tabular}                                                         & Строка              & ---                                                           \\ \hline
			allergies       & Аллергия  & Строка              & ---                                                           \\ \hline
			blood\_type       & Группа крови  & Строка              & ---                                                           \\ \hline
			vaccination       & Вакцинация  & Строка              & ---                                                           \\ \hline
			patientID             &   \begin{tabular}[c]{@{}c@{}}Идентификатор\\ пациента\end{tabular} & Целое число              & \begin{tabular}[c]{@{}c@{}}Внешний ключ,\\ не нулевой\end{tabular}                                                           \\ \hline
		\end{tabular}
	\end{center}
\end{table}


\begin{table}[!ht]
	\begin{center}
		\label{table:users}
		\caption{Описание таблицы Users}
		\begin{tabular}{|c|c|c|c|}
			\hline
			\textbf{Атрибут} & \textbf{Назначение}                                               & \textbf{Тип данных} & \textbf{Ограничение}                                                 \\ \hline
			id                & \begin{tabular}[c]{@{}c@{}}Идентификатор\\ пользователя\end{tabular}     & Целое число         & \begin{tabular}[c]{@{}c@{}}Первичный ключ,\\ не нулевой\end{tabular} \\ \hline
			login               & \begin{tabular}[c]{@{}c@{}}Логин \\пользователя\end{tabular}                                                         & Строка              &    Не нулевой                                                        \\ \hline
			password       & \begin{tabular}[c]{@{}c@{}}Пароль \\пользователя\end{tabular}                                                         & Строка              &    Не нулевой                                                        \\ \hline
			role       & \begin{tabular}[c]{@{}c@{}}Роль \\пользователя\end{tabular}                                                         & Целое число              &    Не нулевой                                                        \\ \hline
			patientID             &   \begin{tabular}[c]{@{}c@{}}Идентификатор\\ пациента\end{tabular} & Целое число              & Внешний ключ                                                           \\ \hline
			doctorID             &   \begin{tabular}[c]{@{}c@{}}Идентификатор\\ врача\end{tabular} & Целое число              & Внешний ключ                                                           \\ \hline
		\end{tabular}
	\end{center}
\end{table}

\begin{table}[!ht]
	\begin{center}
		\label{table:appointments}
		\caption{Описание таблицы Appointments}
		\begin{tabular}{|c|c|c|c|}
			\hline
			\textbf{Атрибут} & \textbf{Назначение}                                               & \textbf{Тип данных} & \textbf{Ограничение}                                                 \\ \hline
			id                & \begin{tabular}[c]{@{}c@{}}Идентификатор\\ записи\end{tabular}     & Целое число         & \begin{tabular}[c]{@{}c@{}}Первичный ключ,\\ не нулевой\end{tabular} \\ \hline
			doctorID               &  \begin{tabular}[c]{@{}c@{}}Идентификатор\\ врача\end{tabular}                                                         & Целое число              & \begin{tabular}[c]{@{}c@{}}Внешний ключ,\\ не нулевой\end{tabular}                                                            \\ \hline
			patientID       &  \begin{tabular}[c]{@{}c@{}}Идентификатор\\ пациента\end{tabular}  & Целое число              & \begin{tabular}[c]{@{}c@{}}Внешний ключ,\\ не нулевой\end{tabular}                                                            \\ \hline
			datetime        &  Дата и время записи & timestamp              &  Не нулевой                                                           \\ \hline
		\end{tabular}
	\end{center}
\end{table}


\begin{table}[!ht]
	\begin{center}
		\caption{Описание таблицы Timetable}
		\label{table:timetable}
		\begin{tabular}{|c|c|c|c|}
			\hline
			\textbf{Атрибут} & \textbf{Назначение}                                               & \textbf{Тип данных} & \textbf{Ограничение}                                                 \\ \hline
			id                & \begin{tabular}[c]{@{}c@{}}Идентификатор\\ записи\end{tabular}     & Целое число         & \begin{tabular}[c]{@{}c@{}}Первичный ключ,\\ не нулевой\end{tabular} \\ \hline
			doctorID               &      \begin{tabular}[c]{@{}c@{}}Идентификатор\\ врача\end{tabular}                                                      & Целое число              & \begin{tabular}[c]{@{}c@{}}Внешний ключ,\\ не нулевой\end{tabular}                                                            \\ \hline
			officeID               &      \begin{tabular}[c]{@{}c@{}}Идентификатор\\ кабинета\end{tabular}                                                      & Целое число              & \begin{tabular}[c]{@{}c@{}}Внешний ключ,\\ не нулевой\end{tabular}                                                            \\ \hline
			workDays             &  Рабочий день & Целое число              & Не нулевой                                                          \\ \hline
		\end{tabular}
	\end{center}
\end{table}


\clearpage
\section{Хранимая функция базы данных}

Хранимая функция -- это набор SQL-выражений, которые выполняют некоторую операцию и возвращают значение.

Разработанная функция $add\_appointment()$ предназначена для создания новой записи приема пациента к врачу в определенное время.
Она принимает 3 параметра: идентификатор врача, идентификатор пациента и дату и время записи на прием.
В теле функции выполняется проверка существования врача и пациента с такими идентификаторами.
В случае успеха создается новая запись в таблице Appointments с заданными значениями полей.

Схема алгоритма работы хранимой функции приведена на рисунке \ref{img:db_func}.

\img{110mm}{db_func}{Схема алгоритма работы хранимой функции создания записи к врачу}

\clearpage
\section{Роли уровня базы данных}
Для каждой из ролей сервиса, описанных ранее, необходимо создать роль уровня базы данных. 

\begin{itemize}
	\item Врач может просмотреть информацию о своих записях, а также создать медицинскую карту пациента, просмотреть информацию о ней и вносить изменения в медицинскую карту определенного пациента.
	Следовательно, врач должен иметь возможность выполнять операции SELECT в таблице Appointments и операции SELECT, UPDATE и INSERT в таблице MedicalHistories, а также врач может выполнять операцию SELECT в таблице Patients, поскольку ему нужно знать данные пациента, которого нужно осмотреть.
	\item Пациент может записаться на прием к определенному врачу в определенное время, просмотреть информацию о всех своих записях, а также изменить информацию о них или отменить их.
	Следовательно, пациент должен иметь возможность выполнять операции SELECT, INSERT, UPDATE и DELETE в таблице Appointments.
	Поскольку пациент должен знать имена врачей, к которым он записывается, а также дни их работы, то у него должна быть возможность получить информацию из таблиц Doctors и Timetable с помощью операции SELECT.
	\item Администратор может создать, удалить или изменить запись, просмотреть информацию об определенной записи по идентификатору, получить список всех врачей, список всех пациентов, внести изменения в любую таблицу или добавить/удалить информацию в таблицах.
	Следовательно, администратор должен иметь возможность выполнять операции SELECT, INSERT, UPDATE и DELETE во всех существующих таблицах без ограничений.
\end{itemize}

\section*{Вывод}
В данном разделе была спроектирована база данных, описаны таблицы и хранимая функция, а также выделены роли уровня базы данных.
