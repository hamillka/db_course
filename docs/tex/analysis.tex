\chapter{Аналитическая часть}

\section{Анализ существующих решений}
Рассмотрим популярные существующие решения, такие как: Поликлиника.ру, Семейный доктор, Семейная поликлиника.
В таблице~\ref{table:existing} представлены результаты сравнительного анализа существующих решений.

Критерии сравнительного анализа:
\begin{itemize}
	\item а -- возможность записи на прием с помощью сайта;
	\item б -- возможность отмены записи с помощью сайта;
	\item в -- возможность изменить дату и время записи;
\end{itemize}

\begin{table}[!ht]
	\caption{Сравнительный анализ существующих решений}
	\label{table:existing}
	\begin{tabular}{|c|c|c|c|c|}
		\hline
		\textbf{} & \textbf{Поликлиника.ру} & \textbf{\begin{tabular}[c]{@{}c@{}}Семейный\\ доктор\end{tabular}} & \textbf{\begin{tabular}[c]{@{}c@{}}Семейная \\ поликлиника\end{tabular}} & \textbf{\begin{tabular}[c]{@{}c@{}}Предлагаемое \\ решение\end{tabular}} \\ \hline
		а         & +                       & +                                                                  & -                                                                        & +                                                                        \\ \hline
		б         & +                       & -                                                                  & +                                                                        & +                                                                        \\ \hline
		в         & -                       & -                                                                  & -                                                                        & +                                                                        \\ \hline
	\end{tabular}
\end{table}

Из всех представленных в таблице~\ref{table:existing} аналогичных решений сеть <<Поликлиника.ру>> является самой полнофункциональной. 
Но данная сеть клиник не позволяет пользователю изменить время приема к врачу через сайт без подтверждения действия через оператора. 
Разрабатываемое решение устраняет недостатки этого сервиса.

\section{Формализация задачи}
В данной курсовой работе необходимо реализовать базу данных сети медицинских клиник, которая содержит в себе информацию о врачах, пациентах и их медицинских картах, а также сведения о записях пациентов к врачам.
Помимо реализации базы данных необходимо разработать веб-приложение, позволяющее пользователю взаимодействовать с ней.

\section{Формализация данных}

База данных сети клиник должна включать в себя информацию о следующих категориях данных: 
\begin{itemize}
	\item пациент;
	\item медицинская карта пациента;
	\item врач;
	\item филиал клиники;
	\item кабинет филиала;
	\item расписание врача;
	\item запись пациента к врачу.
\end{itemize}

В таблице~\ref{table:categories} представлены категории данных и информация о них.

\begin{table}[!ht]
	\caption{Категории данных и информация о них}
	\label{table:categories}
	\begin{center}
		\begin{tabular}{|c|c|}
			\hline
			\textbf{Категория}      & \textbf{Сведения}                                                                      \\ \hline
			Врач                    &  \begin{tabular}[c]{@{}c@{}}ФИО, номер телефона, почта,\\ специализация \end{tabular}                                                            \\ \hline
			Пациент                 & \begin{tabular}[c]{@{}c@{}}ФИО, номер телефона, почта, \\ номер страховки\end{tabular} \\ \hline
			Медицинская карта       & \begin{tabular}[c]{@{}c@{}}Хронические заболевания, аллергия,\\группа крови, вакцинация \end{tabular}                                                      \\ \hline
			Филиал                  & Название, номер телефона, адрес                                                        \\ \hline
			Кабинет                 & Номер, этаж, филиал                                                                    \\ \hline
			Расписание врача        & Врач, кабинет, дни недели                                                              \\ \hline
			Запись пациента к врачу & Врач, пациент, дата и время                                                            \\ \hline
		\end{tabular}
	\end{center}
\end{table}
 
 \newpage
 На рисунке~\ref{img:er} представлена ER-диаграмма БД в нотации Чена.
 
 \img{145mm}{er}{ER-диаграмма в нотации Чена}
 
 \clearpage
\section{Формализация пользовательских ролей}

В процессе проектирования выделены следующие пользовательские роли: 
\begin{itemize}
	\item пациент;
	\item врач;
	\item администратор.
\end{itemize}

Пациент может записаться к врачу на прием, отменить или изменить свою запись, просмотреть свою медицинскую карту, а также получить информацию о своих записях к врачу.
На рисунке~\ref{img:usecasepatient} приведена диаграмма прецедентов для пациента.
\img{55mm}{usecasepatient}{Диаграмма прецедентов для пациента}

Врач может получить список записанных к нему пациентов, создать медицинскую карту любого пациента, просмотреть ее, а также внести в нее изменения.
На рисунке~\ref{img:usecasedoctor} приведена диаграмма прецедентов для врача.
\img{40mm}{usecasedoctor}{Диаграмма прецедентов для врача}

\newpage
Роль администратора расширяет роли врача и пациента: он может выполнять все доступные действия с записями, а также получить список всех врачей и пациентов.
На рисунке~\ref{img:usecaseadmin} приведена диаграмма прецедентов для администратора.

\img{55mm}{usecaseadmin}{Диаграмма прецедентов для администратора}

\clearpage
\section{Модели баз данных}

База данных -- это упорядоченный набор структурированной информации или данных, которые обычно хранятся в электронном виде в компьютерной системе~\cite{db_opr}.
База данных обычно управляется системой управления базами данных (СУБД). 
Данные вместе с СУБД, а также приложения, которые с ними связаны, называются системой баз данных, или, для краткости, просто базой данных. 

Модели баз данных бывают дореляционные, реляционные и постреляционные.

\subsection{Дореляционные модели баз данных}

К дореляционным моделям БД относят иерархическую и сетевую модель, а также инвертированные списки.

Организация данных в СУБД \textbf{иерархического типа}~\cite{db_models} определяется в следующих терминах:
\begin{enumerate}
	\item Атрибут (элемент данных) -- наименьшая единица структуры данных.
	Обычно каждому элементу при описании базы данных присваивается уникальное имя. 
	По этому имени к нему обращаются при обработке. 
	Элемент данных также часто называют полем.
	\item Запись -- именованная совокупность атрибутов.
	Использование записей позволяет за одно обращение к базе получить некоторую логически связанную совокупность данных. 
	Именно записи изменяются, добавляются и удаляются. 
	Тип записи определяется составом ее атрибутов. 
	Экземпляр записи -- конкретная запись с конкретным значением элементов.
	\item Групповое отношение -- иерархическое отношение между записями двух типов. 
	Родительская запись (владелец группового отношения) называется исходной записью, а дочерние записи (члены группового отношения) -- подчиненными. 
	Иерархическая база данных может хранить только такие древовидные структуры.
\end{enumerate}

Корневая запись каждого дерева обязательно должна содержать ключ с уникальным значением. 
Ключи некорневых записей должны иметь уникальное значение только в рамках группового отношения. 
Каждая запись идентифицируется полным сцепленным ключом, под которым понимается совокупность ключей всех записей от корневой по иерархическому пути.

\textbf{Сетевая модель} данных определяется в тех же терминах, что и иерархическая~\cite{db_models}. 
Она состоит из множества записей, которые могут быть владельцами или членами групповых отношений. 
Связь между записью-владельцем и записью-членом также имеет вид 1:N.

Основное различие этих моделей состоит в том, что в сетевой модели запись может быть членом более чем одного группового отношения. 
Согласно этой модели каждое групповое отношение именуется и проводится различие между его типом и экземпляром. 
Тип группового отношения задается его именем и определяет свойства, общие для всех экземпляров данного типа. 
Экземпляр группового отношения представляется записью-владельцем и множеством (возможно пустым) подчиненных записей.
При этом имеется ограничение: экземпляр записи не может быть членом двух экземпляров групповых отношений одного типа.

\textbf{Инвертированный список} -- это организованный в вид списка специального вида вторичный индекс, позволяющий получить всю совокупность указателей на элементы данных (записи), которым соответствует заданное значение ключа индексации~\cite{inv_list}.

Организация доступа к данным на основе инвертированных списков используется практически во всех современных реляционных СУБД, но в этих системах пользователи не имеют непосредственного доступа к инвертированным спискам (индексам).

\subsection{Реляционные модели баз данных}

Реляционная модель представляет собой совокупность данных, состоящую из набора двумерных таблиц~\cite{relat}. 
В теории множеств таблице соответствует термин отношение, физическим представлением которого является таблица.
Реляционная модель является удобной и наиболее привычной формой представления данных.

При табличной организации данных отсутствует иерархия элементов. 
Строки и столбцы могут быть просмотрены в любом порядке, поэтому высока гибкость выбора любого подмножества элементов в строках и столбцах.

Любая таблица в реляционной базе состоит из строк, которые называют записями, и столбцов, которые называют полями. 
На пересечении строк и столбцов находятся конкретные значения данных.

%Реляционная модель объединяет данные в таблицы, где каждая строка представляет собой отдельную запись, а каждый столбец состоит из атрибутов, содержащих значения. 
%Табличный формат позволяет легко устанавливать связи между точками данных и получать доступ к информации любым необходимым способом, не реорганизовывая данные~\cite{relatmodel}.

\subsection{Постреляционные модели баз данных}

Постреляционная модель данных -- это расширенная реляционная модель, в которой отменено требование атомарности атрибутов~\cite{postrelat}.

Она использует трехмерные структуры и позволяет хранить в полях таблицы другие таблицы.
Это расширяет возможности по описанию сложных объектов реального мира.

В качестве языка запросов используется несколько расширенный SQL. 
Он позволяет извлекать сложные объекты из одной таблицы без операций соединения.



%Постреляционная модель данных представляет собой расширенную реляционную модель, снимающую ограничение неделимости данных, хранящихся в записях таблиц. 
%Постреляционная модель данных допускает многозначные поля -- поля, значения которых состоят из подзначений. 
%Набор значений многозначных полей считается самостоятельной таблицей, встроенной в основную таблицу~\cite{postrelatmodel}.


\section*{Вывод}
В данном разделе были рассмотрены существующие решения веб-приложений медицинских клиник и предложено собственное решение, удовлетворяющее выдвинутым критериям. 
Также формализована задача и описаны пользовательские роли и модели баз данных.
Для реализации собственного решения была выбрана реляционная модель, поскольку она представляет сущности в виде отношений. 
Такой способ представления данных является наиболее подходящим для решения поставленных задач.
